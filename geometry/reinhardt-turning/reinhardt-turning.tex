% Thomas Hales
% Nov 8, 2016
% arXiv version, Mar 3, 2017.
% Reinhardt Conjecture 2, control theory
%[12pt]

% hexameral -> hexagonally symmetric
% domain -> disk

% http://english.stackexchange.com/questions/79847/parametrise-or-parameterise-a-curve
% parameter spell, using parametrization.


\documentclass{article}

\usepackage{graphicx}
\usepackage{amsthm,xcolor}
\usepackage{amsfonts}
\usepackage{amsmath}
\usepackage{amscd}
\usepackage{amssymb}
\usepackage{alltt}
\usepackage{url}
\usepackage{ellipsis}
% 
%tikz graphics
%\usepackage{xcolor} % to remove color.
\usepackage{tikz} % 
\usetikzlibrary{chains,shapes,arrows,%
 trees,matrix,positioning,decorations,fadings}
%\usepackage[framemethod=TikZ]{mdframed}

\def\tikzfig#1#2#3{%
\begin{figure}[htb]%
  \centering
\begin{tikzpicture}#3
\end{tikzpicture}
  \caption{#2}
  \label{fig:#1}%
\end{figure}%
}
\def\smalldot#1{\draw[fill=black] (#1) %
 node [inner sep=1.3pt,shape=circle,fill=black] {}}
\newtheorem{theorem}{Theorem}[subsection]
\newtheorem{lemma}[theorem]{Lemma}
\newtheorem{corollary}[theorem]{Corollary}
\newtheorem{proposition}[theorem]{Proposition}
\newtheorem{definition}[theorem]{Definition}
\newtheorem{example}[theorem]{Example}
\newtheorem{working}[theorem]{Working Hypothesis}

\theoremstyle{remark}
\newtheorem{remark}[equation]{Remark}%[subsection]


\newcommand{\ring}[1]{\mathbb{#1}}
\newcommand{\op}[1]{\hbox{#1}}
\newcommand{\f}[1]{\frac{1}{#1}}
\newcommand{\ang}[1]{\left\langle{#1}\right\rangle}
\def\error{e__r__r__o__r}
\def\cong{\error}
\def\sl{\mathfrak{sl}_2(\ring{R})}
\def\SL{\op{SL}_2(\ring{R})}
\def\SO{\op{SO}_2(\ring{R})}
\def\h{\mathfrak h}
\def\hstar{{\mathfrak h}^\star}
\def\Mstar{M^\star}
\def\D{\ring{D}}
\def\Hcost{H_{cost}}
\def\Hlie{H_{Lie}}
\def\Hh{H_{\h}}
\def\DR{D_{min}}
\def\tmax{t_{max}}

\newcommand\Lsing{\Lambda_{sing}}
\newcommand\lsing{\lambda_{sing}}
\newcommand\ee[1]{e_{#1}^*}
\newcommand{\partials}[2]{\frac{\partial #1}{\partial #2}}

%\newcommand\XX[1]{[XX fix: #1]}
%\newcommand\FIGXX{{\tt [XX Insert figure here]}}
%\newcommand{\rif}[1]{\ref{#1}{\tt-#1}}
%\newcommand{\cat}[1]{\cite{#1}{\tt-#1}}
%\newcommand{\libel}[1]{\label{#1}{\tt(#1)~}}


\title{The Turning number for critical points of the Reinhardt optimal control problem}
\author{Thomas C. Hales}  
\date{} 
\date{February 26, 2020}

\begin{document}

\maketitle


\begin{abstract} [Tentative. This hasn't been proved.] We define the turning number of curve
satisfying the optimal control problem attached to the Reinhardt conjecture.
 \end{abstract}

\parskip=0.8\baselineskip
\baselineskip=1.05\baselineskip

\newenvironment{blockquote}{%
  \par%
  \medskip%
  \baselineskip=0.7\baselineskip%
  \leftskip=2em\rightskip=2em%
  \noindent\ignorespaces}{%
  \par\medskip}

\section{Introduction}

In 1934, Reinhardt conjectured that the shape of centrally symmetric
body in the plane whose densest lattice packing has the smallest
density is a smoothed octagon.  

This conjecture has been formulated as a problem in optimal control theory.
The Pontrjagin maximum principle gives a set of conditions defining critical
points of the optimal control problem.  We show that an integer called the turning number can
be attached to each critical point, provided the it avoids the singular locus
of the problem.

Let $\lambda=(g,X,\Lambda,\lambda,\nu)$ be a trajectory of the
Reinhardt optimal control that avoids the singular locus and that
satisfies Pointrjagin's conditions.    

The group $\SL$ acts on the upper half-plane $H$ by linear fractional transformations.
Let $i = \sqrt{-1}\in H$. 
The element $g$ is a path in $\SL$ that satisfies $g(t_0) =I$, $g(t_1)=R\in SO(2)$.
Thus, $g(t_0).i = g(t_1).i = i$.  In particular $t\mapsto g(t).i$ traces a closed curve in $H$.
We show that the turning number of this closed curve can be defined.  

This is not immediate, because
the curve is not known to be differentiable. 
The curve $g$ is differentiable with a Lipschitz continuous derivative.   
Writing $f:\SL\to H$, $h\mapsto h.i$, if $v \in T_{g(t)} \SL$, we have
$Df(v) \in T_{g(t).i} H$.  If $Df(v)$ is nonzero, then the curve $g(t).i$ is differentiable at $t$.

Recall $g' = g X$.   Write $g = b r$, using the Iwasawa decomposition,
with $b \in B$ the upper triangular subgroup and $r \in SO(2)$.
Write $r = \exp(J\theta)$.
Then
\begin{equation}\label{eqn:XJ}
g^{-1} g' = X= r^{-1} b^{-1} b' r + r^{-1} r' = r^{-1} b^{-1} b' r + J \theta'.
\end{equation}



\begin{lemma}
 $Df(v) = 0$ iff $X = J$. 
\end{lemma}

We remark that $X = J$ iff $z = i$.     That is, the closed curve is not regular at $t$ iff the derivative (view in $H$) is
the point $i\in H$, under the usual identification of Lie algebra elements with $H$.

\begin{proof}
Assume that $X=J$.  Then (\ref{eqn:XJ}) implies that $b^{-1} b'$ is a multiple of $J$, which implies that $b'=0$.
We have $g.i = b.i$.  So $(g.i)' = (b.i)'  =0$.    

Conversely, if $(b.i)'=0$, then $b^{-1} b'=0$, so that $X = J\theta'$.  The unit speed condition and clockwise rotation
give $\theta'=1$.  Thus $X=J$.
\end{proof}

It follows from the lemma that there is a well-defined turning number of the closed curve except possibly if the
derivative $z\in H$ is a trajectory passing through $i\in H$.

We have shown that a Pointrjagin optimal trajectory that does not pass through the singular locus is a bang-bang
solution with finitely many switching points.  

We first consider the case that the trajectory $z$ does not have a switching point at $i$.
By symmetry of the control triangle, and the explicit solutions, we may assume that the solution is
\[
x = -m + c_0 e^{\alpha t},\quad y = c_0 \alpha e^{\alpha t}.
\]
The condition that this passes through $i$ at $t=0$ gives 
\[
m= c_0=1/\sqrt{3},\quad  \alpha = 1/c_0.
\]
Solving the ODE for the trajectory $g$ in a power series gives
\[
g(t).i  = -\sqrt{3} t^2 + i (1 + t^2) \mod t^3. 
\] 
This is not regular at $t=0$. (PLOT)
However, if we perturb $\alpha$ to $\alpha = 1/c_0(1+ s)$, for small $s$, we obtain
\[
(-2 s t  -\sqrt{3} (1+s)t^2)  \quad + i (1 + 2 s) t^2 \mod (s^2,t^3).
\]











\bibliography{refs} 
\bibliographystyle{alpha}


\end{document}
