% Thomas Hales
% The Formal Proof of the Kepler Conjecture in Retrospective
% February 17, 2020. started.


%\documentclass[runningheads]{llncs}
%\documentclass{llncs}
\documentclass{amsart}

\usepackage{graphicx}
\usepackage{pdfpages}
%\usepackage{amsthm}
\usepackage{xcolor}
\usepackage{amsfonts}
\usepackage{amsmath}
\usepackage{amscd}
\usepackage{amssymb}
\usepackage{alltt}
\usepackage{url}
\usepackage{ellipsis}

% For code snippets
\usepackage{isabelle}
\usepackage{isabellesym}
\usepackage{isabelletags}
\usepackage{comment}
\usepackage{pdfsetup}
\usepackage{railsetup}
\usepackage{wasysym}

% Bibliography
% Change style to the requested style of the template
%\usepackage[backend=bibtex, style=numeric]{biblatex}
%\addbibresource{refs.bib}

% 
%tikz graphics
%\usepackage{xcolor} % to remove color.
\usepackage{tikz} % 
\usetikzlibrary{chains,shapes,arrows,%
 trees,matrix,positioning,decorations}
%\usepackage[framemethod=TikZ]{mdframed}

\def\tikzfig#1#2#3{%
\begin{figure}[htb]%
  \centering
\begin{tikzpicture}#3
\end{tikzpicture}
  \caption{#2}
  \label{fig:#1}%
\end{figure}%
}
\def\smalldot#1{\draw[fill=black] (#1) %
 node [inner sep=1.3pt,shape=circle,fill=black] {}}

%\newtheorem{theorem}{Theorem}[subsection]
%\newtheorem{lemma}[theorem]{Lemma}
%\newtheorem{corollary}[theorem]{Corollary}
%\newtheorem{proposition}[theorem]{Proposition}
%\newtheorem{definition}[theorem]{Definition}

\newcommand{\ring}[1]{\mathbb{#1}}
\newcommand{\op}[1]{\hbox{#1}}

\title{The Formal Proof of the Kepler Conjecture in Retrospective}
\author{Thomas Hales}
\date{}
%\institute{University of Pittsburgh}
\date{February 20, 2020}

\begin{document}

\maketitle

\begin{abstract} 
  The Kepler conjecture asserts that no packing of congruent balls in
  Euclidean three-space has density greater than that of the
  face-centered cubic packing.  In 1998, Samuel Ferguson and Thomas
  Hales announced a computer-assisted proof of this conjecture.  Long
  delays in the refereeing process sparked a project to give a formal
  proof of the Kepler conjecture.  This article gives an overview of
  the formal proof of the Kepler Conjecture.
\end{abstract}

\parskip=0.8\baselineskip
\baselineskip=1.05\baselineskip

\newenvironment{blockquote}{%
  \par%
  \medskip%
  \baselineskip=0.7\baselineskip%
  \leftskip=2em\rightskip=2em%
  \noindent\ignorespaces}{%
  \par\medskip}

\section{Introduction}

In the late 16th century, Walter Raleigh asked his assistant Thomas
Harriot to compute the number of cannonballs piled in a pyramid (in
defense against the Spanish Armada).  Thomas Harriot obtained a
general formula (as the binomial coefficient $\choose{a}{b}$) for the
number of congruent balls piled in a pyramid of any size $k$ in any
dimension $d$.  In two dimensions, Harriot's formula reduces to the
formula $k(k+1)/2$ for triangular numbers.  In three dimensions,
Harriot's formula reduces to a formula known from antiquity in a
Sanskrit source.  The cannonball pyramid is known in chemistry as the
face-centered cubic packing.

Harriot was an early proponent of the atomic theory, and he viewed
atoms as small balls -- which might almost be viewed as miniature
cannonballs.  Harriot and Kepler shared an interest in optics, and
they corresponded during the first decade of the 17th century.
Harriot pushed the atomic theory in their correspondence, but Kepler
remained skeptical.

However, one evening crossing the Charles bridge in Prague, as it
started to snow, Kepler started to contemplate why snowflakes have six
sides.  His contemplations led to a booklet ``The Six-Cornered
Snowflake,'' which can be described as an early essay in
crystallography, explaining observable symmetries in nature by
arrangements of invisibly small particles.  In this booklet, published
in 1611, Kepler describes the face-centered cubic packing and asserts
that it will be the ``tightest possible, so that in no other
arrangement can more pellets be stuffed into the same space.'' The
notion of density was suggested to him by the tightly packed seeds of
a pomegranite, and the face-centered cubic packing was suggested to
him by the structure of honeycomb cells.

Kepler's assertion has come to be known as the Kepler conjecture.
Hilbert made the Kepler conjecture part of the 18th problem, one among
the influential list of problems he proposed at the international
congress in Paris in 1900.

The two-dimensional analogue of the Kepler conjecture asserts that the
densest packing of congruent circular disks in the plane is achieved
by the hexagonal lattice packing.  A. Thue is often credited with the
proof, although the first careful proofs do not appear until much
later.

The first detailed strategy to prove the Kepler conjecture was
formulated by L. Fejes T\'oth in 1953.  He was also the first to
suggest a computer-assisted proof.

Over the years, there were some notorious false claims of a proof.
Buckminster Fuller (the creator of the geodesic dome) claimed a proof,
but his claim lacked any substance.  Around 1990, a Berkeley professor
W.-Y. Hsiang claimed a proof.  In rebuttal to his claimed proof, in a
heated debate, I published a paper containing explicit counterexamples
to his work.  For a few years in the early 1990s, the research area
became toxic, until the debate was eventually settled in my favor.

Samuel Ferguson and I annouced a solution to the Kepler conjecture in
August 1998.  Our solution was contained in a series of preprints
posted to the ArXiv.  Because of the conjecture's notorious history, I
had hoped that publication of our papers would be swift. This was not
to be.  A panel of 12 referees was assigned the review.  Because of
lingering doubts among the referees, the full publication of the proof
did not occur until 2006, nearly eight years after submission.

During those years in delay, growing out of my frustration, in an
effort to bypass the referees, I launched a project to give a formal
proof of the Kepler conjecture.  Although the project was born out of
frustration, in truth, it was also
shaped by a broader vision of the central importance of
computer-assisted mathematics in years to come.  In a large group
collaboration, the formal proof of the Kepler conjecture was completed
in 2014.

\section{Why HOL Light?}

I made the decision as a non-expert.  HOL Light had decisive
advantages: a well-developed theory of real analysis, no boundaries
between programming and proof script-writing, etc.

\section{Alignment between formal and informal}

(Euler example)

\section{A next generation proof}

The formalization of the Kepler conjecture was far from optimal.  This
final section points towards a second generation proof.

\subsection{nonlinear inequalities}
It was understood from the very beginning of the formalization of the
Kepler conjecture that the largest challenge would be the
formalization of the nonlinear inequalities.  Specifically,
floating-point arithmetic is slower by orders of magnitude when
executed as logical rules in HOL Light than in a native processor.

The proof of the Kepler conjecture relies on about a thousand
nonlinear inequalities over the real numbers.  These inequalities
are proved by computer using interval arithmetic. The inequalities
have the general form 
\begin{equation}\label{eqn:ineq}
\forall x \in D,\quad f_1(x) < 0 \lor f_2(x) < 0 \lor \ldots f_k(x) < 0.
\end{equation}
Each function $f_i : R_i \to \ring{R}$ is defined on a subset $R_i$ of the
domain $D\subset \ring{R}^n$.
The inequality (\ref{eqn:ineq}) means more precisely that 
at each point $x\in D$, there exists $1\le i\le k$ such that
$x \in R_i$ and $f_i(x) < 0$. 

The domain $D$ is a product of 
compact intervals $[a,b]$, and $n$ is small (usually $n\le 6$).  
The dimension $n=6$ of the domain $D$ occurs naturally, because a
simplex in three dimensions is determined by its six edges.

There was an extraordinary amount of freedom in the creation
of a finite set of nonlinear inequalities that collectively imply
the Kepler conjecture.  In no sense is the current collection
optimal.  Further research is justified.



Tricks used.

Tricks not used.



Isabelle-HOL-Light interface.

Multiple list libraries and uniform style.

DRY Alignment.

Search - Zumkeller.

Tools - Libraries (Harrison), standalone nonlinear (Solovyev), linear programming (Solovyev).

Tactic styles (no distinction between Ocaml code and proof scripts).

Overlap of definitions (list libraries Isabelle, HOL Light, Coq).



\section{Concluding Remarks}

Some have expressed an ambition to use artificial intelligence to
\emph{solve math} in 
the coming years.  To the ambitious, we offer the more modest
challenge of using AI to develop a next-generation proof of the Kepler
conjecture.  The current proof is now over 22 years old.


\newpage

%\printbibliography

\bibliography{refs} 
\bibliographystyle{alpha}

\end{document}

