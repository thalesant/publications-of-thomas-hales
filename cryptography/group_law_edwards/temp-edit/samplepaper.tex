% This is samplepaper.tex, a sample chapter demonstrating the
% LLNCS macro package for Springer Computer Science proceedings;
% Version 2.20 of 2017/10/04
%
\documentclass[runningheads]{llncs}
%
\usepackage{graphicx}
\usepackage{pdfpages}
\usepackage{amsmath}
\usepackage{amssymb}

% For code snippets
\usepackage{isabelle}
\usepackage{isabellesym}
\usepackage{isabelletags}
\usepackage{comment}
\usepackage{pdfsetup}
\usepackage{railsetup}
\usepackage{wasysym}

% Bibliography
% Change style to the requested style of the template
\usepackage[backend=bibtex, style=numeric]{biblatex}
\addbibresource{references.bib}

% Used for displaying a sample figure. If possible, figure files should
% be included in EPS format.
%
% If you use the hyperref package, please uncomment the following line
% to display URLs in blue roman font according to Springer's eBook style:
% \renewcommand\UrlFont{\color{blue}\rmfamily}

\begin{document}

\section{Formalization in Isabelle/HOL}

In this section, we describe the proof implementation in Isabelle/HOL. It uses two different locales: one for the affine and one for the projective case.

Let $k$ be the underlying curve field. $k$ is introduced as the type class \textit{field} with the assumption that $2 \neq 0$ (characteristic different from $2$). This is not included in the simplification set, but used when needed during the proof.   

\subsection{Affine Edwards curves}

The proof starts fixing the curve parameters $c,d \in k$. The group addition $\oplus_0$ can be written as in figure \ref{fig:1}. One could remark that in Isabelle's division ring theory, dividing by zero is defined as zero. This has no impact on the proof, but some simplifications are clearer.

\begin{figure}
	{%
	
\begin{isabellebody}%

add\ {\isacharcolon}{\isacharcolon}\ {\isacharprime}a\ {\isasymtimes}\ {\isacharprime}a\ {\isasymRightarrow}\ {\isacharprime}a\ {\isasymtimes}\ {\isacharprime}a\ {\isasymRightarrow}\ {\isacharprime}a\ {\isasymtimes}\ {\isacharprime}a\ \isanewline
add\ {\isacharparenleft}x\isactrlsub {\isadigit{1}}{\isacharcomma}y\isactrlsub {\isadigit{1}}{\isacharparenright}\ {\isacharparenleft}x\isactrlsub {\isadigit{2}}{\isacharcomma}y\isactrlsub {\isadigit{2}}{\isacharparenright}\ {\isacharequal}
{\isacharparenleft}{\isacharparenleft}x\isactrlsub {\isadigit{1}}{\isacharasterisk}x\isactrlsub {\isadigit{2}}\ {\isacharminus}\ c{\isacharasterisk}y\isactrlsub {\isadigit{1}}{\isacharasterisk}y\isactrlsub {\isadigit{2}}{\isacharparenright}\ div {\isacharparenleft}{\isadigit{1}}{\isacharminus}d{\isacharasterisk}x\isactrlsub {\isadigit{1}}{\isacharasterisk}y\isactrlsub {\isadigit{1}}{\isacharasterisk}x\isactrlsub {\isadigit{2}}{\isacharasterisk}y\isactrlsub {\isadigit{2}}{\isacharparenright}{\isacharcomma}\ \isanewline
\ \ \ \ \ \ \ \ \ \ \ \ \ \ \ \ \ \ \ \ \ \ \ {\isacharparenleft}x\isactrlsub {\isadigit{1}}{\isacharasterisk}y\isactrlsub {\isadigit{2}}{\isacharplus}y\isactrlsub {\isadigit{1}}{\isacharasterisk}x\isactrlsub {\isadigit{2}}{\isacharparenright}\ div\ {\isacharparenleft}{\isadigit{1}}{\isacharplus}d{\isacharasterisk}x\isactrlsub {\isadigit{1}}{\isacharasterisk}y\isactrlsub {\isadigit{1}}{\isacharasterisk}x\isactrlsub {\isadigit{2}}{\isacharasterisk}y\isactrlsub {\isadigit{2}}{\isacharparenright}{\isacharparenright}
\end{isabellebody}%



%:%file=~/Isabelle/curves_safe/Hales.thy%:%
%:%10=1%:%
%:%11=1%:%
%:%12=2%:%
%:%13=3%:%
%:%18=3%:%
%:%21=4%:%
%:%22=5%:%
%:%23=5%:%
%:%30=8%:%
%:%40=10%:%
%:%41=10%:%
%:%42=11%:%
%:%43=12%:%
%:%44=13%:%
%:%45=13%:%
%:%46=14%:%
%:%47=15%:%
%:%48=16%:%
%:%49=17%:%
%:%50=17%:%
%:%51=18%:%
%:%53=20%:%
%:%54=21%:%
%:%55=22%:%
%:%56=22%:%
%:%57=23%:%
%:%58=24%:%
%:%59=25%:%
%:%60=25%:%
%:%61=26%:%
%:%62=27%:%
%:%63=28%:%
%:%64=28%:%
%:%65=29%:%
%:%66=30%:%
%:%67=31%:%
%:%68=32%:%
%:%69=32%:%
%:%70=33%:%
%:%71=34%:%
%:%72=35%:%
%:%73=35%:%
%:%74=36%:%
%:%75=37%:%
%:%76=38%:%
%:%77=39%:%
%:%78=40%:%
%:%79=41%:%
%:%80=42%:%
%:%81=43%:%
%:%88=44%:%
%:%89=44%:%
%:%90=45%:%
%:%91=45%:%
%:%92=46%:%
%:%93=46%:%
%:%94=47%:%
%:%95=47%:%
%:%96=48%:%
%:%97=48%:%
%:%99=50%:%
%:%100=51%:%
%:%101=51%:%
%:%103=53%:%
%:%104=54%:%
%:%105=54%:%
%:%106=55%:%
%:%107=55%:%
%:%108=56%:%
%:%109=56%:%
%:%110=57%:%
%:%111=57%:%
%:%112=58%:%
%:%113=59%:%
%:%114=59%:%
%:%115=60%:%
%:%116=60%:%
%:%117=60%:%
%:%118=61%:%
%:%119=61%:%
%:%120=62%:%
%:%121=62%:%
%:%122=62%:%
%:%123=63%:%
%:%124=63%:%
%:%125=64%:%
%:%126=64%:%
%:%127=64%:%
%:%128=65%:%
%:%129=65%:%
%:%130=66%:%
%:%131=66%:%
%:%132=66%:%
%:%133=67%:%
%:%134=68%:%
%:%135=68%:%
%:%136=69%:%
%:%137=69%:%
%:%138=69%:%
%:%139=70%:%
%:%140=71%:%
%:%141=71%:%
%:%148=78%:%
%:%149=79%:%
%:%150=79%:%
%:%151=80%:%
%:%152=80%:%
%:%153=81%:%
%:%154=81%:%
%:%155=82%:%
%:%156=82%:%
%:%157=83%:%
%:%158=83%:%
%:%159=84%:%
%:%160=85%:%
%:%161=85%:%
%:%162=86%:%
%:%167=91%:%
%:%168=92%:%
%:%169=92%:%
%:%170=93%:%
%:%171=93%:%
%:%172=94%:%
%:%173=94%:%
%:%174=95%:%
%:%175=95%:%
%:%176=96%:%
%:%177=96%:%
%:%178=97%:%
%:%179=98%:%
%:%180=98%:%
%:%181=99%:%
%:%182=99%:%
%:%183=100%:%
%:%184=100%:%
%:%185=101%:%
%:%186=101%:%
%:%187=102%:%
%:%188=102%:%
%:%189=103%:%
%:%190=103%:%
%:%191=104%:%
%:%192=104%:%
%:%193=105%:%
%:%194=105%:%
%:%195=106%:%
%:%196=107%:%
%:%197=107%:%
%:%198=108%:%
%:%199=109%:%
%:%200=109%:%
%:%201=109%:%
%:%202=110%:%
%:%203=110%:%
%:%204=111%:%
%:%205=112%:%
%:%211=112%:%
%:%214=113%:%
%:%215=114%:%
%:%223=118%:%
%:%233=120%:%
%:%234=120%:%
%:%235=121%:%
%:%236=122%:%
%:%237=123%:%
%:%238=124%:%
%:%239=125%:%
%:%240=126%:%
%:%241=127%:%
%:%242=127%:%
%:%243=128%:%
%:%245=130%:%
%:%246=131%:%
%:%247=132%:%
%:%248=132%:%
%:%249=133%:%
%:%250=134%:%
%:%251=134%:%
%:%252=135%:%
%:%253=136%:%
%:%254=137%:%
%:%255=138%:%
%:%256=139%:%
%:%257=140%:%
%:%258=140%:%
%:%259=141%:%
%:%260=142%:%
%:%261=142%:%
%:%262=143%:%
%:%263=144%:%
%:%264=144%:%
%:%265=145%:%
%:%266=146%:%
%:%267=147%:%
%:%268=147%:%
%:%269=148%:%
%:%270=149%:%
%:%271=149%:%
%:%272=150%:%
%:%273=151%:%
%:%274=151%:%
%:%275=152%:%
%:%278=155%:%
%:%279=156%:%
%:%280=157%:%
%:%281=158%:%
%:%282=158%:%
%:%283=159%:%
%:%284=160%:%
%:%285=161%:%
%:%288=162%:%
%:%292=162%:%
%:%298=162%:%
%:%301=163%:%
%:%302=164%:%
%:%303=164%:%
%:%304=165%:%
%:%305=166%:%
%:%306=166%:%
%:%307=167%:%
%:%308=168%:%
%:%309=169%:%
%:%310=170%:%
%:%313=171%:%
%:%317=171%:%
%:%323=171%:%
%:%326=172%:%
%:%327=173%:%
%:%328=173%:%
%:%330=173%:%
%:%334=173%:%
%:%335=173%:%
%:%336=173%:%
%:%343=173%:%
%:%344=174%:%
%:%345=175%:%
%:%346=175%:%
%:%347=176%:%
%:%349=178%:%
%:%350=179%:%
%:%351=180%:%
%:%352=180%:%
%:%353=181%:%
%:%355=183%:%
%:%356=184%:%
%:%357=185%:%
%:%358=185%:%
%:%359=186%:%
%:%360=187%:%
%:%361=188%:%
%:%362=189%:%
%:%363=189%:%
%:%364=190%:%
%:%365=191%:%
%:%366=192%:%
%:%367=192%:%
%:%368=193%:%
%:%369=194%:%
%:%370=195%:%
%:%379=204%:%
%:%380=205%:%
%:%383=206%:%
%:%387=206%:%
%:%393=206%:%
%:%396=207%:%
%:%397=208%:%
%:%398=208%:%
%:%400=208%:%
%:%404=208%:%
%:%405=208%:%
%:%406=208%:%
%:%413=208%:%
%:%414=209%:%
%:%415=210%:%
%:%416=210%:%
%:%417=211%:%
%:%418=212%:%
%:%419=213%:%
%:%420=214%:%
%:%421=215%:%
%:%422=216%:%
%:%423=217%:%
%:%424=217%:%
%:%427=218%:%
%:%432=219%:%}
	\caption{Definition of $\oplus_0$ in Isabelle/HOL}
	\label{fig:1}
\end{figure}

Most of the proofs in this section are straight-forward. The only difficulty was to combine the Mathematica computations of [2] and their certificates, originally given in HOL Light, into a single process using the Isabelle distribution.

In figure \ref{fig:2}, we show an excerpt of the proof of associativity. We use the following abbreviations: \[e_i = x_i^2 + c * y_i^2 - 1 - d * x_i^2 * y_i^2 \] where $e_i = 0$, since the involved points lie on the curve and: \[\text{gxpoly} = ((p_1 \oplus_0 p_2) \oplus_0 p_3 - p_1 \oplus_0 (p_2 \oplus p_3))_1*\Delta_x\] which stands for a normalized version of the associativity law removing denominators. Since the points $p_i$ are assumed to be addable, $\Delta_x \neq 0$. As a consequence, the property stated in figure \ref{fig:2} immediately implies that associativity holds in the first component of the addition.


\begin{figure}
	{%
	
\begin{isabellebody}%

\ \ \isacommand{have}\isamarkupfalse%
\ {\isachardoublequoteopen}{\isasymexists}\ r{\isadigit{1}}\ r{\isadigit{2}}\ r{\isadigit{3}}{\isachardot}\ gxpoly\ {\isacharequal}\ r{\isadigit{1}}\ {\isacharasterisk}\ e{\isadigit{1}}\ {\isacharplus}\ r{\isadigit{2}}\ {\isacharasterisk}\ e{\isadigit{2}}\ {\isacharplus}\ r{\isadigit{3}}\ {\isacharasterisk}\ e{\isadigit{3}}{\isachardoublequoteclose}\isanewline
\ \ \ \ \isacommand{unfolding}\isamarkupfalse%
\ gxpoly{\isacharunderscore}def\ g\isactrlsub x{\isacharunderscore}def\ Delta\isactrlsub x{\isacharunderscore}def\ \isanewline
\ \ \ \ \isacommand{apply}\isamarkupfalse%
{\isacharparenleft}simp\ add{\isacharcolon}\ assms{\isacharparenleft}{\isadigit{1}}{\isacharcomma}{\isadigit{2}}{\isacharparenright}{\isacharparenright}\isanewline
\ \ \ \ \isacommand{apply}\isamarkupfalse%
{\isacharparenleft}rewrite\ \isakeyword{in}\ {\isachardoublequoteopen}{\isacharunderscore}\ {\isacharslash}\ {\isasymhole}{\isachardoublequoteclose}\ delta{\isacharunderscore}minus{\isacharunderscore}def{\isacharbrackleft}symmetric{\isacharbrackright}{\isacharparenright}{\isacharplus}\isanewline
\ \ \ \ \isacommand{apply}\isamarkupfalse%
{\isacharparenleft}simp\ add{\isacharcolon}\ divide{\isacharunderscore}simps\ assms{\isacharparenleft}{\isadigit{9}}{\isacharcomma}{\isadigit{1}}{\isadigit{1}}{\isacharparenright}{\isacharparenright}\isanewline
\ \ \ \ \isacommand{apply}\isamarkupfalse%
{\isacharparenleft}rewrite\ left{\isacharunderscore}diff{\isacharunderscore}distrib{\isacharparenright}\isanewline
\ \ \ \ \isacommand{apply}\isamarkupfalse%
{\isacharparenleft}simp\ add{\isacharcolon}\ simp{\isadigit{1}}gx\ simp{\isadigit{2}}gx{\isacharparenright}\isanewline
\ \ \ \ \isacommand{unfolding}\isamarkupfalse%
\ delta{\isacharunderscore}plus{\isacharunderscore}def\ delta{\isacharunderscore}minus{\isacharunderscore}def\isanewline
\ \ \ \ \ \ \ \ \ \ \ \ \ \ e{\isadigit{1}}{\isacharunderscore}def\ e{\isadigit{2}}{\isacharunderscore}def\ e{\isadigit{3}}{\isacharunderscore}def\ e{\isacharunderscore}def\isanewline
\ \ \ \ \isacommand{by}\isamarkupfalse%
\ algebra\isanewline

\end{isabellebody}%



%:%file=~/Isabelle/curves_safe/Hales.thy%:%
%:%10=1%:%
%:%11=1%:%
%:%12=2%:%
%:%13=3%:%
%:%18=3%:%
%:%21=4%:%
%:%22=5%:%
%:%23=5%:%
%:%30=8%:%
%:%40=10%:%
%:%41=10%:%
%:%42=11%:%
%:%43=12%:%
%:%44=13%:%
%:%45=13%:%
%:%46=14%:%
%:%47=15%:%
%:%48=16%:%
%:%49=17%:%
%:%50=17%:%
%:%51=18%:%
%:%53=20%:%
%:%54=21%:%
%:%55=22%:%
%:%56=22%:%
%:%57=23%:%
%:%58=24%:%
%:%59=25%:%
%:%60=25%:%
%:%61=26%:%
%:%62=27%:%
%:%63=28%:%
%:%64=28%:%
%:%65=29%:%
%:%66=30%:%
%:%67=31%:%
%:%68=32%:%
%:%69=32%:%
%:%70=33%:%
%:%71=34%:%
%:%72=35%:%
%:%73=35%:%
%:%74=36%:%
%:%75=37%:%
%:%76=38%:%
%:%77=39%:%
%:%78=40%:%
%:%79=41%:%
%:%80=42%:%
%:%81=43%:%
%:%88=44%:%
%:%89=44%:%
%:%90=45%:%
%:%91=45%:%
%:%92=46%:%
%:%93=46%:%
%:%94=47%:%
%:%95=47%:%
%:%96=48%:%
%:%97=48%:%
%:%99=50%:%
%:%100=51%:%
%:%101=51%:%
%:%103=53%:%
%:%104=54%:%
%:%105=54%:%
%:%106=55%:%
%:%107=55%:%
%:%108=56%:%
%:%109=56%:%
%:%110=57%:%
%:%111=57%:%
%:%112=58%:%
%:%113=59%:%
%:%114=59%:%
%:%115=60%:%
%:%116=60%:%
%:%117=60%:%
%:%118=61%:%
%:%119=61%:%
%:%120=62%:%
%:%121=62%:%
%:%122=62%:%
%:%123=63%:%
%:%124=63%:%
%:%125=64%:%
%:%126=64%:%
%:%127=64%:%
%:%128=65%:%
%:%129=65%:%
%:%130=66%:%
%:%131=66%:%
%:%132=66%:%
%:%133=67%:%
%:%134=68%:%
%:%135=68%:%
%:%136=69%:%
%:%137=69%:%
%:%138=69%:%
%:%139=70%:%
%:%140=71%:%
%:%141=71%:%
%:%148=78%:%
%:%149=79%:%
%:%150=79%:%
%:%151=80%:%
%:%152=80%:%
%:%153=81%:%
%:%154=81%:%
%:%155=82%:%
%:%156=82%:%
%:%157=83%:%
%:%158=83%:%
%:%159=84%:%
%:%160=85%:%
%:%161=85%:%
%:%162=86%:%
%:%167=91%:%
%:%168=92%:%
%:%169=92%:%
%:%170=93%:%
%:%171=93%:%
%:%172=94%:%
%:%173=94%:%
%:%174=95%:%
%:%175=95%:%
%:%176=96%:%
%:%177=96%:%
%:%178=97%:%
%:%179=98%:%
%:%180=98%:%
%:%181=99%:%
%:%182=99%:%
%:%183=100%:%
%:%184=100%:%
%:%185=101%:%
%:%186=101%:%
%:%187=102%:%
%:%188=102%:%
%:%189=103%:%
%:%190=103%:%
%:%191=104%:%
%:%192=104%:%
%:%193=105%:%
%:%194=105%:%
%:%195=106%:%
%:%196=107%:%
%:%197=107%:%
%:%198=108%:%
%:%199=109%:%
%:%200=109%:%
%:%201=109%:%
%:%202=110%:%
%:%203=110%:%
%:%204=111%:%
%:%205=112%:%
%:%211=112%:%
%:%214=113%:%
%:%215=114%:%
%:%223=118%:%
%:%233=120%:%
%:%234=120%:%
%:%235=121%:%
%:%236=122%:%
%:%237=123%:%
%:%238=124%:%
%:%239=125%:%
%:%240=126%:%
%:%241=127%:%
%:%242=127%:%
%:%243=128%:%
%:%245=130%:%
%:%246=131%:%
%:%247=132%:%
%:%248=132%:%
%:%249=133%:%
%:%250=134%:%
%:%251=134%:%
%:%252=135%:%
%:%253=136%:%
%:%254=137%:%
%:%255=138%:%
%:%256=139%:%
%:%257=140%:%
%:%258=140%:%
%:%259=141%:%
%:%260=142%:%
%:%261=142%:%
%:%262=143%:%
%:%263=144%:%
%:%264=144%:%
%:%265=145%:%
%:%266=146%:%
%:%267=147%:%
%:%268=147%:%
%:%269=148%:%
%:%270=149%:%
%:%271=149%:%
%:%272=150%:%
%:%273=151%:%
%:%274=151%:%
%:%275=152%:%
%:%278=155%:%
%:%279=156%:%
%:%280=157%:%
%:%281=158%:%
%:%282=158%:%
%:%283=159%:%
%:%284=160%:%
%:%285=161%:%
%:%288=162%:%
%:%292=162%:%
%:%298=162%:%
%:%301=163%:%
%:%302=164%:%
%:%303=164%:%
%:%304=165%:%
%:%305=166%:%
%:%306=166%:%
%:%307=167%:%
%:%308=168%:%
%:%309=169%:%
%:%310=170%:%
%:%313=171%:%
%:%317=171%:%
%:%323=171%:%
%:%326=172%:%
%:%327=173%:%
%:%328=173%:%
%:%330=173%:%
%:%334=173%:%
%:%335=173%:%
%:%336=173%:%
%:%343=173%:%
%:%344=174%:%
%:%345=175%:%
%:%346=175%:%
%:%347=176%:%
%:%349=178%:%
%:%350=179%:%
%:%351=180%:%
%:%352=180%:%
%:%353=181%:%
%:%355=183%:%
%:%356=184%:%
%:%357=185%:%
%:%358=185%:%
%:%359=186%:%
%:%360=187%:%
%:%361=188%:%
%:%362=189%:%
%:%363=189%:%
%:%364=190%:%
%:%365=191%:%
%:%366=192%:%
%:%367=192%:%
%:%368=193%:%
%:%369=194%:%
%:%370=195%:%
%:%379=204%:%
%:%380=205%:%
%:%383=206%:%
%:%387=206%:%
%:%393=206%:%
%:%396=207%:%
%:%397=208%:%
%:%398=208%:%
%:%400=208%:%
%:%404=208%:%
%:%405=208%:%
%:%406=208%:%
%:%413=208%:%
%:%414=209%:%
%:%415=210%:%
%:%416=210%:%
%:%417=211%:%
%:%418=212%:%
%:%419=213%:%
%:%420=214%:%
%:%421=215%:%
%:%422=216%:%
%:%423=217%:%
%:%424=217%:%
%:%427=218%:%
%:%432=219%:%}
	\caption{An excerpt of the proof of associativity}
	\label{fig:2}
\end{figure}

Briefly, the proof unfolds the relevant definitions and then normalizes to eliminate denominators. The remaining terms of $\Delta_x$ are then distributed on each addend. The unfolding and normalization for these addends is repeated in the sub-lemmas \textit{simp1gx} and \textit{simp2gx}. Finally, a polynomial expression is obtained and the \textit{algebra} method can prove it. Note that no computation was required from an external tool. 

For the normalization of terms, the \textit{rewrite} tactic \cite{noschinskipattern} was chosen. It allows to modify a goal in several locations (specified with a pattern) and using several rewrite rules. Specifying that rewriting happens in the denominators is sufficient for our needs.

For proving the resulting polynomial expression, the \textit{algebra} proof method \cite{chaieb2008automated} is used. There are two applications of the method. First, it addresses universally quantified formulas with equations or disequations as atoms. In this case, the method is only complete for algebraically closed fields \cite{wenzel2019isabelle}. Second, for $e_i(x_1,\ldots,x_n), p_{ij}(x_1,\ldots,x_n),a_i(x_1,\ldots,x_n) \in R[x_1,\ldots,x_n]$ where $R$ is a commutative ring, the method can solve the formula:  \[\; \forall \overline{x}. \bigwedge_{i = 1}^n e_i(\overline{x}) = 0 \to \exists \overline{y}. \bigwedge_{i = 1}^k \sum_{j = 1}^m p_{ij}(\overline{x}) y_j = a_i(\overline{x})\] and the method is complete for properties that hold over all commutative rings with unit \cite{harrison2007automating}. 

\subsection{Projective Edwards curves}

Following the change of variables performed in section 4.1 of \cite{hales2016group}, it is assumed that $c = 1$ and $d = t^2$ where $t \neq -1,0,1$. The resulting proof is more challenging. In the following, some key insights are emphasised. 

\subsubsection{Gr\"{o}ebner basis}

The proof of dichotomy requires solving particular instances of the ideal membership problem \cite{hales2016group}. In our practice, we found that the \textit{algebra} method requires some hints to solve some of these problems. For instance, in one occasion for the goal: \[\exists q_1 \, q_2 \, q_3 \, q_4.
y_0^2 - x_1^2 = 
q_1 e(x_0,y_0) +
q_2 e(x_1,y_1) +
q_3 \delta' +
q_4 \delta_{-} \]  it was necessary to multiply $y_0^2 - x_1^2$ by the invertible elements $2,x_0$ and $y_0$, in order for the proof to succeed. In another sub-case, it was necessary to strengthen the hypothesis $\delta_{+} = 0$ to $\delta_{-} \neq 0$. So, with some help, \textit{algebra} is able to solve of the required ideal membership problems.

\subsubsection{Definition of the group addition}

We defined the addition in three stages. This is convenient for some theorems, like covering. First, we define the addition on projective points (figure \ref{fig3}). Then, we add two classes of points by applying the basic addition to any pair of points coming from each class. Finally, we apply the gluing relation and obtain as a result a set of classes with a unique element, which is then defined as the resulting class (figure \ref{fig4}).

\begin{figure}
{%
	
\begin{isabellebody}%

\isacommand{type{\isacharunderscore}synonym}\isamarkupfalse%
\ {\isacharparenleft}{\isacharprime}b{\isacharparenright}\ ppoint\ {\isacharequal}\ {\isacartoucheopen}{\isacharparenleft}{\isacharparenleft}{\isacharprime}b\ {\isasymtimes}\ {\isacharprime}b{\isacharparenright}\ {\isasymtimes}\ bit{\isacharparenright}{\isacartoucheclose}\isanewline

p{\isacharunderscore}add\ {\isacharcolon}{\isacharcolon}\ {\isacharprime}a\ ppoint\ {\isasymRightarrow}\ {\isacharprime}a\ ppoint\ {\isasymRightarrow}\ {\isacharprime}a\ ppoint\ \isakeyword{where}\ \isanewline
\ \ p{\isacharunderscore}add\ {\isacharparenleft}{\isacharparenleft}x\isactrlsub {\isadigit{1}}{\isacharcomma}\ y\isactrlsub {\isadigit{1}}{\isacharparenright}{\isacharcomma}\ l{\isacharparenright}\ {\isacharparenleft}{\isacharparenleft}x\isactrlsub {\isadigit{2}}{\isacharcomma}\ y\isactrlsub {\isadigit{2}}{\isacharparenright}{\isacharcomma}\ j{\isacharparenright}\ {\isacharequal}\ {\isacharparenleft}add\ {\isacharparenleft}x\isactrlsub {\isadigit{1}}{\isacharcomma}\ y\isactrlsub {\isadigit{1}}{\isacharparenright}\ {\isacharparenleft}x\isactrlsub {\isadigit{2}}{\isacharcomma}\ y\isactrlsub {\isadigit{2}}{\isacharparenright}{\isacharcomma}\ l{\isacharplus}j{\isacharparenright}\isanewline
\ \isakeyword{if}\ delta\ x\isactrlsub {\isadigit{1}}\ y\isactrlsub {\isadigit{1}}\ x\isactrlsub {\isadigit{2}}\ y\isactrlsub {\isadigit{2}}\ {\isasymnoteq}\ {\isadigit{0}}\ {\isasymand}\ {\isacharparenleft}x\isactrlsub {\isadigit{1}}{\isacharcomma}\ y\isactrlsub {\isadigit{1}}{\isacharparenright}\ {\isasymin}\ e{\isacharprime}{\isacharunderscore}aff\ {\isasymand}\ {\isacharparenleft}x\isactrlsub {\isadigit{2}}{\isacharcomma}\ y\isactrlsub {\isadigit{2}}{\isacharparenright}\ {\isasymin}\ e{\isacharprime}{\isacharunderscore}aff\ \isanewline
{\isacharbar}\ p{\isacharunderscore}add\ {\isacharparenleft}{\isacharparenleft}x\isactrlsub {\isadigit{1}}{\isacharcomma}\ y\isactrlsub {\isadigit{1}}{\isacharparenright}{\isacharcomma}\ l{\isacharparenright}\ {\isacharparenleft}{\isacharparenleft}x\isactrlsub {\isadigit{2}}{\isacharcomma}\ y\isactrlsub {\isadigit{2}}{\isacharparenright}{\isacharcomma}\ j{\isacharparenright}\ {\isacharequal}\ {\isacharparenleft}ext{\isacharunderscore}add\ {\isacharparenleft}x\isactrlsub {\isadigit{1}}{\isacharcomma}\ y\isactrlsub {\isadigit{1}}{\isacharparenright}\ {\isacharparenleft}x\isactrlsub {\isadigit{2}}{\isacharcomma}\ y\isactrlsub {\isadigit{2}}{\isacharparenright}{\isacharcomma}\ l{\isacharplus}j{\isacharparenright}\isanewline
\ \isakeyword{if}\ delta{\isacharprime}\ x\isactrlsub {\isadigit{1}}\ y\isactrlsub {\isadigit{1}}\ x\isactrlsub {\isadigit{2}}\ y\isactrlsub {\isadigit{2}}\ {\isasymnoteq}\ {\isadigit{0}}\ {\isasymand}\ {\isacharparenleft}x\isactrlsub {\isadigit{1}}{\isacharcomma}\ y\isactrlsub {\isadigit{1}}{\isacharparenright}\ {\isasymin}\ e{\isacharprime}{\isacharunderscore}aff\ {\isasymand}\ {\isacharparenleft}x\isactrlsub {\isadigit{2}}{\isacharcomma}\ y\isactrlsub {\isadigit{2}}{\isacharparenright}\ {\isasymin}\ e{\isacharprime}{\isacharunderscore}aff

\end{isabellebody}%



%:%file=~/Isabelle/curves_safe/Hales.thy%:%
%:%10=1%:%
%:%11=1%:%
%:%12=2%:%
%:%13=3%:%
%:%18=3%:%
%:%21=4%:%
%:%22=5%:%
%:%23=5%:%
%:%30=8%:%
%:%40=10%:%
%:%41=10%:%
%:%42=11%:%
%:%43=12%:%
%:%44=13%:%
%:%45=13%:%
%:%46=14%:%
%:%47=15%:%
%:%48=16%:%
%:%49=17%:%
%:%50=17%:%
%:%51=18%:%
%:%53=20%:%
%:%54=21%:%
%:%55=22%:%
%:%56=22%:%
%:%57=23%:%
%:%58=24%:%
%:%59=25%:%
%:%60=25%:%
%:%61=26%:%
%:%62=27%:%
%:%63=28%:%
%:%64=28%:%
%:%65=29%:%
%:%66=30%:%
%:%67=31%:%
%:%68=32%:%
%:%69=32%:%
%:%70=33%:%
%:%71=34%:%
%:%72=35%:%
%:%73=35%:%
%:%74=36%:%
%:%75=37%:%
%:%76=38%:%
%:%77=39%:%
%:%78=40%:%
%:%79=41%:%
%:%80=42%:%
%:%81=43%:%
%:%88=44%:%
%:%89=44%:%
%:%90=45%:%
%:%91=45%:%
%:%92=46%:%
%:%93=46%:%
%:%94=47%:%
%:%95=47%:%
%:%96=48%:%
%:%97=48%:%
%:%99=50%:%
%:%100=51%:%
%:%101=51%:%
%:%103=53%:%
%:%104=54%:%
%:%105=54%:%
%:%106=55%:%
%:%107=55%:%
%:%108=56%:%
%:%109=56%:%
%:%110=57%:%
%:%111=57%:%
%:%112=58%:%
%:%113=59%:%
%:%114=59%:%
%:%115=60%:%
%:%116=60%:%
%:%117=60%:%
%:%118=61%:%
%:%119=61%:%
%:%120=62%:%
%:%121=62%:%
%:%122=62%:%
%:%123=63%:%
%:%124=63%:%
%:%125=64%:%
%:%126=64%:%
%:%127=64%:%
%:%128=65%:%
%:%129=65%:%
%:%130=66%:%
%:%131=66%:%
%:%132=66%:%
%:%133=67%:%
%:%134=68%:%
%:%135=68%:%
%:%136=69%:%
%:%137=69%:%
%:%138=69%:%
%:%139=70%:%
%:%140=71%:%
%:%141=71%:%
%:%148=78%:%
%:%149=79%:%
%:%150=79%:%
%:%151=80%:%
%:%152=80%:%
%:%153=81%:%
%:%154=81%:%
%:%155=82%:%
%:%156=82%:%
%:%157=83%:%
%:%158=83%:%
%:%159=84%:%
%:%160=85%:%
%:%161=85%:%
%:%162=86%:%
%:%167=91%:%
%:%168=92%:%
%:%169=92%:%
%:%170=93%:%
%:%171=93%:%
%:%172=94%:%
%:%173=94%:%
%:%174=95%:%
%:%175=95%:%
%:%176=96%:%
%:%177=96%:%
%:%178=97%:%
%:%179=98%:%
%:%180=98%:%
%:%181=99%:%
%:%182=99%:%
%:%183=100%:%
%:%184=100%:%
%:%185=101%:%
%:%186=101%:%
%:%187=102%:%
%:%188=102%:%
%:%189=103%:%
%:%190=103%:%
%:%191=104%:%
%:%192=104%:%
%:%193=105%:%
%:%194=105%:%
%:%195=106%:%
%:%196=107%:%
%:%197=107%:%
%:%198=108%:%
%:%199=109%:%
%:%200=109%:%
%:%201=109%:%
%:%202=110%:%
%:%203=110%:%
%:%204=111%:%
%:%205=112%:%
%:%211=112%:%
%:%214=113%:%
%:%215=114%:%
%:%223=118%:%
%:%233=120%:%
%:%234=120%:%
%:%235=121%:%
%:%236=122%:%
%:%237=123%:%
%:%238=124%:%
%:%239=125%:%
%:%240=126%:%
%:%241=127%:%
%:%242=127%:%
%:%243=128%:%
%:%245=130%:%
%:%246=131%:%
%:%247=132%:%
%:%248=132%:%
%:%249=133%:%
%:%250=134%:%
%:%251=134%:%
%:%252=135%:%
%:%253=136%:%
%:%254=137%:%
%:%255=138%:%
%:%256=139%:%
%:%257=140%:%
%:%258=140%:%
%:%259=141%:%
%:%260=142%:%
%:%261=142%:%
%:%262=143%:%
%:%263=144%:%
%:%264=144%:%
%:%265=145%:%
%:%266=146%:%
%:%267=147%:%
%:%268=147%:%
%:%269=148%:%
%:%270=149%:%
%:%271=149%:%
%:%272=150%:%
%:%273=151%:%
%:%274=151%:%
%:%275=152%:%
%:%278=155%:%
%:%279=156%:%
%:%280=157%:%
%:%281=158%:%
%:%282=158%:%
%:%283=159%:%
%:%284=160%:%
%:%285=161%:%
%:%288=162%:%
%:%292=162%:%
%:%298=162%:%
%:%301=163%:%
%:%302=164%:%
%:%303=164%:%
%:%304=165%:%
%:%305=166%:%
%:%306=166%:%
%:%307=167%:%
%:%308=168%:%
%:%309=169%:%
%:%310=170%:%
%:%313=171%:%
%:%317=171%:%
%:%323=171%:%
%:%326=172%:%
%:%327=173%:%
%:%328=173%:%
%:%330=173%:%
%:%334=173%:%
%:%335=173%:%
%:%336=173%:%
%:%343=173%:%
%:%344=174%:%
%:%345=175%:%
%:%346=175%:%
%:%347=176%:%
%:%349=178%:%
%:%350=179%:%
%:%351=180%:%
%:%352=180%:%
%:%353=181%:%
%:%355=183%:%
%:%356=184%:%
%:%357=185%:%
%:%358=185%:%
%:%359=186%:%
%:%360=187%:%
%:%361=188%:%
%:%362=189%:%
%:%363=189%:%
%:%364=190%:%
%:%365=191%:%
%:%366=192%:%
%:%367=192%:%
%:%368=193%:%
%:%369=194%:%
%:%370=195%:%
%:%379=204%:%
%:%380=205%:%
%:%383=206%:%
%:%387=206%:%
%:%393=206%:%
%:%396=207%:%
%:%397=208%:%
%:%398=208%:%
%:%400=208%:%
%:%404=208%:%
%:%405=208%:%
%:%406=208%:%
%:%413=208%:%
%:%414=209%:%
%:%415=210%:%
%:%416=210%:%
%:%417=211%:%
%:%418=212%:%
%:%419=213%:%
%:%420=214%:%
%:%421=215%:%
%:%422=216%:%
%:%423=217%:%
%:%424=217%:%
%:%427=218%:%
%:%432=219%:%}
\caption{Definition of $\oplus$ on points}
\label{fig3}
\end{figure}
\begin{figure}
	{%
	
\begin{isabellebody}%
\isacommand{type{\isacharunderscore}synonym}\isamarkupfalse%
\ {\isacharparenleft}{\isacharprime}b{\isacharparenright}\ pclass\ {\isacharequal}\ {\isacartoucheopen}{\isacharparenleft}{\isacharprime}b{\isacharparenright}\ ppoint\ set{\isacartoucheclose} \\

proj{\isacharunderscore}add{\isacharunderscore}class\ {\isacharcolon}{\isacharcolon}\ {\isacharparenleft}{\isacharprime}a{\isacharparenright}\ pclass\ {\isasymRightarrow}\ {\isacharparenleft}{\isacharprime}a{\isacharparenright}\ pclass\ {\isasymRightarrow}\ {\isacharparenleft}{\isacharprime}a{\isacharparenright}\ pclass\ set\isanewline
\ \ proj{\isacharunderscore}add{\isacharunderscore}class\ c\isactrlsub {\isadigit{1}}\ c\isactrlsub {\isadigit{2}}\ {\isacharequal}\ \ \ \ \ \ \ \isanewline
\ \ \ \ \ \ \ \ {\isacharparenleft}p{\isacharunderscore}add\ {\isacharbackquote}\ {\isacharbraceleft}{\isacharparenleft}{\isacharparenleft}{\isacharparenleft}x\isactrlsub {\isadigit{1}}{\isacharcomma}\ y\isactrlsub {\isadigit{1}}{\isacharparenright}{\isacharcomma}\ i{\isacharparenright}{\isacharcomma}{\isacharparenleft}{\isacharparenleft}x\isactrlsub {\isadigit{2}}{\isacharcomma}\ y\isactrlsub {\isadigit{2}}{\isacharparenright}{\isacharcomma}\ j{\isacharparenright}{\isacharparenright}{\isachardot} \ \isanewline
\ \ \ \ \ \ \ \ \ \ \ \ \ \ \ \ {\isacharparenleft}{\isacharparenleft}{\isacharparenleft}x\isactrlsub {\isadigit{1}}{\isacharcomma}\ y\isactrlsub {\isadigit{1}}{\isacharparenright}{\isacharcomma}\ i{\isacharparenright}{\isacharcomma}{\isacharparenleft}{\isacharparenleft}x\isactrlsub {\isadigit{2}}{\isacharcomma}\ y\isactrlsub {\isadigit{2}}{\isacharparenright}{\isacharcomma}\ j{\isacharparenright}{\isacharparenright}\ {\isasymin}\ c\isactrlsub {\isadigit{1}}\ {\isasymtimes}\ c\isactrlsub {\isadigit{2}}\ {\isasymand}\ \isanewline
\ \ \ \ \ \ \ \ \ \ \ \ \ \ \ \ {\isacharparenleft}{\isacharparenleft}x\isactrlsub {\isadigit{1}}{\isacharcomma}\ y\isactrlsub {\isadigit{1}}{\isacharparenright}{\isacharcomma}\ {\isacharparenleft}x\isactrlsub {\isadigit{2}}{\isacharcomma}\ y\isactrlsub {\isadigit{2}}{\isacharparenright}{\isacharparenright}\ {\isasymin}\ e{\isacharprime}{\isacharunderscore}aff{\isacharunderscore}{\isadigit{0}}\ {\isasymunion}\ e{\isacharprime}{\isacharunderscore}aff{\isacharunderscore}{\isadigit{1}}{\isacharbraceright}{\isacharparenright}\ {\isacharslash}{\isacharslash}\ gluing\ \isanewline 
\ \isakeyword{if}\ c\isactrlsub {\isadigit{1}}\ {\isasymin}\ e{\isacharunderscore}proj\ \isakeyword{and}\ c\isactrlsub {\isadigit{2}}\ {\isasymin}\ e{\isacharunderscore}proj\ 
\isanewline

\end{isabellebody}%



%:%file=~/Isabelle/curves_safe/Hales.thy%:%
%:%10=1%:%
%:%11=1%:%
%:%12=2%:%
%:%13=3%:%
%:%18=3%:%
%:%21=4%:%
%:%22=5%:%
%:%23=5%:%
%:%30=8%:%
%:%40=10%:%
%:%41=10%:%
%:%42=11%:%
%:%43=12%:%
%:%44=13%:%
%:%45=13%:%
%:%46=14%:%
%:%47=15%:%
%:%48=16%:%
%:%49=17%:%
%:%50=17%:%
%:%51=18%:%
%:%53=20%:%
%:%54=21%:%
%:%55=22%:%
%:%56=22%:%
%:%57=23%:%
%:%58=24%:%
%:%59=25%:%
%:%60=25%:%
%:%61=26%:%
%:%62=27%:%
%:%63=28%:%
%:%64=28%:%
%:%65=29%:%
%:%66=30%:%
%:%67=31%:%
%:%68=32%:%
%:%69=32%:%
%:%70=33%:%
%:%71=34%:%
%:%72=35%:%
%:%73=35%:%
%:%74=36%:%
%:%75=37%:%
%:%76=38%:%
%:%77=39%:%
%:%78=40%:%
%:%79=41%:%
%:%80=42%:%
%:%81=43%:%
%:%88=44%:%
%:%89=44%:%
%:%90=45%:%
%:%91=45%:%
%:%92=46%:%
%:%93=46%:%
%:%94=47%:%
%:%95=47%:%
%:%96=48%:%
%:%97=48%:%
%:%99=50%:%
%:%100=51%:%
%:%101=51%:%
%:%103=53%:%
%:%104=54%:%
%:%105=54%:%
%:%106=55%:%
%:%107=55%:%
%:%108=56%:%
%:%109=56%:%
%:%110=57%:%
%:%111=57%:%
%:%112=58%:%
%:%113=59%:%
%:%114=59%:%
%:%115=60%:%
%:%116=60%:%
%:%117=60%:%
%:%118=61%:%
%:%119=61%:%
%:%120=62%:%
%:%121=62%:%
%:%122=62%:%
%:%123=63%:%
%:%124=63%:%
%:%125=64%:%
%:%126=64%:%
%:%127=64%:%
%:%128=65%:%
%:%129=65%:%
%:%130=66%:%
%:%131=66%:%
%:%132=66%:%
%:%133=67%:%
%:%134=68%:%
%:%135=68%:%
%:%136=69%:%
%:%137=69%:%
%:%138=69%:%
%:%139=70%:%
%:%140=71%:%
%:%141=71%:%
%:%148=78%:%
%:%149=79%:%
%:%150=79%:%
%:%151=80%:%
%:%152=80%:%
%:%153=81%:%
%:%154=81%:%
%:%155=82%:%
%:%156=82%:%
%:%157=83%:%
%:%158=83%:%
%:%159=84%:%
%:%160=85%:%
%:%161=85%:%
%:%162=86%:%
%:%167=91%:%
%:%168=92%:%
%:%169=92%:%
%:%170=93%:%
%:%171=93%:%
%:%172=94%:%
%:%173=94%:%
%:%174=95%:%
%:%175=95%:%
%:%176=96%:%
%:%177=96%:%
%:%178=97%:%
%:%179=98%:%
%:%180=98%:%
%:%181=99%:%
%:%182=99%:%
%:%183=100%:%
%:%184=100%:%
%:%185=101%:%
%:%186=101%:%
%:%187=102%:%
%:%188=102%:%
%:%189=103%:%
%:%190=103%:%
%:%191=104%:%
%:%192=104%:%
%:%193=105%:%
%:%194=105%:%
%:%195=106%:%
%:%196=107%:%
%:%197=107%:%
%:%198=108%:%
%:%199=109%:%
%:%200=109%:%
%:%201=109%:%
%:%202=110%:%
%:%203=110%:%
%:%204=111%:%
%:%205=112%:%
%:%211=112%:%
%:%214=113%:%
%:%215=114%:%
%:%223=118%:%
%:%233=120%:%
%:%234=120%:%
%:%235=121%:%
%:%236=122%:%
%:%237=123%:%
%:%238=124%:%
%:%239=125%:%
%:%240=126%:%
%:%241=127%:%
%:%242=127%:%
%:%243=128%:%
%:%245=130%:%
%:%246=131%:%
%:%247=132%:%
%:%248=132%:%
%:%249=133%:%
%:%250=134%:%
%:%251=134%:%
%:%252=135%:%
%:%253=136%:%
%:%254=137%:%
%:%255=138%:%
%:%256=139%:%
%:%257=140%:%
%:%258=140%:%
%:%259=141%:%
%:%260=142%:%
%:%261=142%:%
%:%262=143%:%
%:%263=144%:%
%:%264=144%:%
%:%265=145%:%
%:%266=146%:%
%:%267=147%:%
%:%268=147%:%
%:%269=148%:%
%:%270=149%:%
%:%271=149%:%
%:%272=150%:%
%:%273=151%:%
%:%274=151%:%
%:%275=152%:%
%:%278=155%:%
%:%279=156%:%
%:%280=157%:%
%:%281=158%:%
%:%282=158%:%
%:%283=159%:%
%:%284=160%:%
%:%285=161%:%
%:%288=162%:%
%:%292=162%:%
%:%298=162%:%
%:%301=163%:%
%:%302=164%:%
%:%303=164%:%
%:%304=165%:%
%:%305=166%:%
%:%306=166%:%
%:%307=167%:%
%:%308=168%:%
%:%309=169%:%
%:%310=170%:%
%:%313=171%:%
%:%317=171%:%
%:%323=171%:%
%:%326=172%:%
%:%327=173%:%
%:%328=173%:%
%:%330=173%:%
%:%334=173%:%
%:%335=173%:%
%:%336=173%:%
%:%343=173%:%
%:%344=174%:%
%:%345=175%:%
%:%346=175%:%
%:%347=176%:%
%:%349=178%:%
%:%350=179%:%
%:%351=180%:%
%:%352=180%:%
%:%353=181%:%
%:%355=183%:%
%:%356=184%:%
%:%357=185%:%
%:%358=185%:%
%:%359=186%:%
%:%360=187%:%
%:%361=188%:%
%:%362=189%:%
%:%363=189%:%
%:%364=190%:%
%:%365=191%:%
%:%366=192%:%
%:%367=192%:%
%:%368=193%:%
%:%369=194%:%
%:%370=195%:%
%:%379=204%:%
%:%380=205%:%
%:%383=206%:%
%:%387=206%:%
%:%393=206%:%
%:%396=207%:%
%:%397=208%:%
%:%398=208%:%
%:%400=208%:%
%:%404=208%:%
%:%405=208%:%
%:%406=208%:%
%:%413=208%:%
%:%414=209%:%
%:%415=210%:%
%:%416=210%:%
%:%417=211%:%
%:%418=212%:%
%:%419=213%:%
%:%420=214%:%
%:%421=215%:%
%:%422=216%:%
%:%423=217%:%
%:%424=217%:%
%:%427=218%:%
%:%432=219%:%}
	{%
	
\begin{isabellebody}%

proj{\isacharunderscore}addition\ c\isactrlsub {\isadigit{1}}\ c\isactrlsub {\isadigit{2}}\ {\isacharequal}\ the{\isacharunderscore}elem\ {\isacharparenleft}proj{\isacharunderscore}add{\isacharunderscore}class\ c\isactrlsub {\isadigit{1}}\ c\isactrlsub {\isadigit{2}}{\isacharparenright}

\end{isabellebody}%



%:%file=~/Isabelle/curves_safe/Hales.thy%:%
%:%10=1%:%
%:%11=1%:%
%:%12=2%:%
%:%13=3%:%
%:%18=3%:%
%:%21=4%:%
%:%22=5%:%
%:%23=5%:%
%:%30=8%:%
%:%40=10%:%
%:%41=10%:%
%:%42=11%:%
%:%43=12%:%
%:%44=13%:%
%:%45=13%:%
%:%46=14%:%
%:%47=15%:%
%:%48=16%:%
%:%49=17%:%
%:%50=17%:%
%:%51=18%:%
%:%53=20%:%
%:%54=21%:%
%:%55=22%:%
%:%56=22%:%
%:%57=23%:%
%:%58=24%:%
%:%59=25%:%
%:%60=25%:%
%:%61=26%:%
%:%62=27%:%
%:%63=28%:%
%:%64=28%:%
%:%65=29%:%
%:%66=30%:%
%:%67=31%:%
%:%68=32%:%
%:%69=32%:%
%:%70=33%:%
%:%71=34%:%
%:%72=35%:%
%:%73=35%:%
%:%74=36%:%
%:%75=37%:%
%:%76=38%:%
%:%77=39%:%
%:%78=40%:%
%:%79=41%:%
%:%80=42%:%
%:%81=43%:%
%:%88=44%:%
%:%89=44%:%
%:%90=45%:%
%:%91=45%:%
%:%92=46%:%
%:%93=46%:%
%:%94=47%:%
%:%95=47%:%
%:%96=48%:%
%:%97=48%:%
%:%99=50%:%
%:%100=51%:%
%:%101=51%:%
%:%103=53%:%
%:%104=54%:%
%:%105=54%:%
%:%106=55%:%
%:%107=55%:%
%:%108=56%:%
%:%109=56%:%
%:%110=57%:%
%:%111=57%:%
%:%112=58%:%
%:%113=59%:%
%:%114=59%:%
%:%115=60%:%
%:%116=60%:%
%:%117=60%:%
%:%118=61%:%
%:%119=61%:%
%:%120=62%:%
%:%121=62%:%
%:%122=62%:%
%:%123=63%:%
%:%124=63%:%
%:%125=64%:%
%:%126=64%:%
%:%127=64%:%
%:%128=65%:%
%:%129=65%:%
%:%130=66%:%
%:%131=66%:%
%:%132=66%:%
%:%133=67%:%
%:%134=68%:%
%:%135=68%:%
%:%136=69%:%
%:%137=69%:%
%:%138=69%:%
%:%139=70%:%
%:%140=71%:%
%:%141=71%:%
%:%148=78%:%
%:%149=79%:%
%:%150=79%:%
%:%151=80%:%
%:%152=80%:%
%:%153=81%:%
%:%154=81%:%
%:%155=82%:%
%:%156=82%:%
%:%157=83%:%
%:%158=83%:%
%:%159=84%:%
%:%160=85%:%
%:%161=85%:%
%:%162=86%:%
%:%167=91%:%
%:%168=92%:%
%:%169=92%:%
%:%170=93%:%
%:%171=93%:%
%:%172=94%:%
%:%173=94%:%
%:%174=95%:%
%:%175=95%:%
%:%176=96%:%
%:%177=96%:%
%:%178=97%:%
%:%179=98%:%
%:%180=98%:%
%:%181=99%:%
%:%182=99%:%
%:%183=100%:%
%:%184=100%:%
%:%185=101%:%
%:%186=101%:%
%:%187=102%:%
%:%188=102%:%
%:%189=103%:%
%:%190=103%:%
%:%191=104%:%
%:%192=104%:%
%:%193=105%:%
%:%194=105%:%
%:%195=106%:%
%:%196=107%:%
%:%197=107%:%
%:%198=108%:%
%:%199=109%:%
%:%200=109%:%
%:%201=109%:%
%:%202=110%:%
%:%203=110%:%
%:%204=111%:%
%:%205=112%:%
%:%211=112%:%
%:%214=113%:%
%:%215=114%:%
%:%223=118%:%
%:%233=120%:%
%:%234=120%:%
%:%235=121%:%
%:%236=122%:%
%:%237=123%:%
%:%238=124%:%
%:%239=125%:%
%:%240=126%:%
%:%241=127%:%
%:%242=127%:%
%:%243=128%:%
%:%245=130%:%
%:%246=131%:%
%:%247=132%:%
%:%248=132%:%
%:%249=133%:%
%:%250=134%:%
%:%251=134%:%
%:%252=135%:%
%:%253=136%:%
%:%254=137%:%
%:%255=138%:%
%:%256=139%:%
%:%257=140%:%
%:%258=140%:%
%:%259=141%:%
%:%260=142%:%
%:%261=142%:%
%:%262=143%:%
%:%263=144%:%
%:%264=144%:%
%:%265=145%:%
%:%266=146%:%
%:%267=147%:%
%:%268=147%:%
%:%269=148%:%
%:%270=149%:%
%:%271=149%:%
%:%272=150%:%
%:%273=151%:%
%:%274=151%:%
%:%275=152%:%
%:%278=155%:%
%:%279=156%:%
%:%280=157%:%
%:%281=158%:%
%:%282=158%:%
%:%283=159%:%
%:%284=160%:%
%:%285=161%:%
%:%288=162%:%
%:%292=162%:%
%:%298=162%:%
%:%301=163%:%
%:%302=164%:%
%:%303=164%:%
%:%304=165%:%
%:%305=166%:%
%:%306=166%:%
%:%307=167%:%
%:%308=168%:%
%:%309=169%:%
%:%310=170%:%
%:%313=171%:%
%:%317=171%:%
%:%323=171%:%
%:%326=172%:%
%:%327=173%:%
%:%328=173%:%
%:%330=173%:%
%:%334=173%:%
%:%335=173%:%
%:%336=173%:%
%:%343=173%:%
%:%344=174%:%
%:%345=175%:%
%:%346=175%:%
%:%347=176%:%
%:%349=178%:%
%:%350=179%:%
%:%351=180%:%
%:%352=180%:%
%:%353=181%:%
%:%355=183%:%
%:%356=184%:%
%:%357=185%:%
%:%358=185%:%
%:%359=186%:%
%:%360=187%:%
%:%361=188%:%
%:%362=189%:%
%:%363=189%:%
%:%364=190%:%
%:%365=191%:%
%:%366=192%:%
%:%367=192%:%
%:%368=193%:%
%:%369=194%:%
%:%370=195%:%
%:%379=204%:%
%:%380=205%:%
%:%383=206%:%
%:%387=206%:%
%:%393=206%:%
%:%396=207%:%
%:%397=208%:%
%:%398=208%:%
%:%400=208%:%
%:%404=208%:%
%:%405=208%:%
%:%406=208%:%
%:%413=208%:%
%:%414=209%:%
%:%415=210%:%
%:%416=210%:%
%:%417=211%:%
%:%418=212%:%
%:%419=213%:%
%:%420=214%:%
%:%421=215%:%
%:%422=216%:%
%:%423=217%:%
%:%424=217%:%
%:%427=218%:%
%:%432=219%:%}
	\caption{Definition of $\oplus$ on classes}
	\label{fig4}
\end{figure}

The definitions use Isabelle's ability to encode partial functions. However, it is possible to obtain an equivalent definition more suitable for execution. In particular, it is easy to compute the gluing relation (see lemmas $\text{e\_proj\_elim\_1}$, $\text{e\_proj\_elim\_2}$ and $\text{e\_proj\_aff}$). 

Finally, since projective addition works with classes, we had to show that its definition does not depend on the representative used. 

\subsubsection{Proof of the associativity law}

During development, we found several relations between $\delta$ expressions (see table \ref{table:1}). While they were proven in order to show associativity, the upper group can rather be used to establish the independence of class representative and the lower group is crucial to establish the associativity law.

\begin{table}
\begin{center}
	\begin{tabular}{ r c l }
 $\delta \; \tau p_1 \; \tau p_2 \neq 0$ & $\implies$ & $\; \delta \; p_1 \; p_2 \neq 0$ \\ 
 $\delta' \; \tau p_1 \; \tau p_2 \neq 0$ & $\implies$ & $\; \delta' \; p_1 \; p_2 \neq 0$ \\ 
 $\delta \; p_1 \; p_2 \neq 0$,$\; \delta \; p_1 \; \tau p_2 \neq 0$ & $\implies$ & $\delta' \; p_1 \; p_2 \neq 0$ \\ 
 $\delta' \; p_1 \; p_2 \neq 0$,$\; \delta' \; p_1 \; \tau p_2 \neq 0$ & $\implies$ & $\delta \; p_1 \;p_2 \neq 0$ \\ 
 \hline
 $\delta' \; (p_1 \oplus_1 p_2) \; \tau i p_2 \neq 0$ & $\implies$ & $\; \delta \; (p_1 \oplus_1 p_2) \; i p_2 \neq 0$ \\ 
 $\delta \; p_1 \; p_2 \neq 0$,$\; \delta \; (p_1 \oplus_0 p_2) \; \tau i p_2 \neq 0$ & $\implies$ & $\; \delta' (p_1 \oplus_0 p_2) \; i p_2 \neq 0$ \\ 
 $\delta \; p_1 \; p_2 \neq 0$,$\; \delta' \; (p_0 \oplus_0 p_1) \; \tau i p_2 \neq 0$ & $\implies$ & $\; \delta \; (p_0 \oplus_0 p_1) \; i p_2 \neq 0$ \\     
 $\delta' \; p_1 \; p_2 \neq 0$,$\; \delta \; (p_0 \oplus_1 p_1) \; \tau i p_2 \neq 0$ & $\implies$ & $\; \delta' \; (p_0 \oplus_1 p_1) \; i p_2 \neq 0$ \\
	\end{tabular}
\end{center}
	\caption{List of $\delta$ relations}
	\label{table:1}
\end{table}
%
In particular, the lower part of the table is fundamental to establish
goal 22 of \cite{hales2016group}, i.e., $([P,0] \oplus [Q,0]) \oplus
[iQ,0] = [P,0]$. Here is how we established it. The proof development
showed that it was necessary to perform a triple dichotomy. The first
dichotomy is performed on $P$, $Q$. The non-addable case was
easy. Therefore, we set $R = P \oplus Q$. On each of the resulting
branches, a dichotomy on $R$, $iQ$ is performed. This time the addable
cases were easy, but the non-addable case required a third dichotomy
on $R,\tau i Q$. The non-addable case was solved using the
no-fixed-point theorem but for the addable subcases the following
expression is obtained: $$([P,0] \oplus [Q,0]) \oplus [\tau i Q,0] =
[(P \oplus Q) \oplus \tau i Q,0]$$ Unfortunately, here we cannot use
associativity since all the involved additions must be defined and
$Q$, $\tau i Q$ cannot be added (lemma
$\text{not\_add\_self}$). Instead, we use the equations on the lower
part of the table and together with the hypothesis of the second
dichotomy, we get a contradiction.

\section{Conclusion}

In this section, we have shown that Isabelle can subsume the process
of defining, computing and certifying intensive algebraic
calculations. The encoding in a proof-assistant allows a better
comprehension of the methods used and helps to clarify its structure.
%
% ---- Bibliography ----
%
% BibTeX users should specify bibliography style 'splncs04'.
% References will then be sorted and formatted in the correct style.
%
% \bibliographystyle{splncs04}
% \bibliography{mybibliography}
\printbibliography
%
\end{document}
