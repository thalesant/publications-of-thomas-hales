\section{The Circle}

The unit circle $\ring{C}_1$ in the complex plane $\ring{C}$ is a
group under complex multiplication, or equivalently under the addition
of angles in polar coordinates:
\begin{equation}\label{eqn:cx}
(x_1,y_1) * (x_2,y_2) = (x_1 x_2 - y_1 y_2,x_1 y_2 + x_2 y_1).
\end{equation}
We write $\iota(x,y) = (x,-y)$ for complex conjugation, the inverse in
$\ring{C}_1$.

We give an unusual interpretation of the group law on the unit circle
that we call {\it hyperbolic addition}.  We consider the family of
hyperbolas in the plane that pass through the point $z=(-1,0)$ and
whose asymptotes are parallel to the coordinate axes.  The equation of
such a hyperbola has the form
\begin{equation}\label{eqn:hyp}
x y + p (x+1) + q y = 0.
\end{equation}
All hyperbolas in this article are assumed to be of this form.  As
special cases (such as $x y=0$), this family includes pairs of lines.
 
Every two points $z_1$ and $z_2$ on the unit circle intersect some
hyperbola within the family.  This incidence condition uniquely
determines $p$ and $q$ when $(-1,0)$, $z_1$ and $z_2$ are not
collinear.  As illustrated in Figure~\ref{fig:fig1}, the hyperbola
meets the unit circle in one additional point $z_3 = (x_3,y_3)$.  The
following remarkable relationship holds among the three points $z_1$,
$z_2$, and $z_3$ on the intersection of the circle and hyperbola.

\begin{lemma}[hyperbolic addition on the circle] \label{lemma:ha} Let
  $z_0=(-1,0)$, $z_1$, $z_2$ and $z_3$ be four distinct points on the
  intersection of the unit circle with a hyperbola in the family (\ref{eqn:hyp}).  
  Then $z_1 z_2 z_3 = 1$ in $\ring{C}_1$.
\end{lemma}

The lemma is a special case of a more general lemma
(Lemma~\ref{lemma:hyperbola}) that is proved later in this article.

This gives a geometric construction of the group law: the product of
the two points $z_1$ and $z_2$ on the unit circle is $\iota(z_3)$.
Rather than starting with the standard formula for addition in
$\ring{C}_1$, we can reverse the process, defining a binary operation
$(\oplus)$ on the circle by setting
\[
z_1 \oplus z_2 = \iota(z_3)
\]
whenever $z_1$, $z_2$, and $z_3$ are related by the circle and
hyperbola construction.  We call the binary operation $\oplus$ {\it
  hyperbolic addition} on the circle.

It might seem that there is no point in reinterpreting complex
multiplication on the unit circle as hyperbolic addition, because they
are actually the same binary operation, and the group $\ring{C}_1$ is
already perfectly well-understood.  However, in the next section, we
will see that hyperbolic addition generalizes in ways that ordinary
multiplication does not.  In this sense, we have found a better
description of the group law on the circle.  The same description
works for elliptic curves!


% Fig1

\tikzfig{fig1}{A unit circle centered at the origin
and hyperbola meet at four points $z_0 = (-1,0)$, $z_1$, $z_2$, and $z_3$,
where $z_1 z_2 z_3 = 1$, which we write alternatively in additive notation 
as $z_1\oplus z_2 = \iota(z_3)$.}{
{
\draw (0,0) circle (1);
\draw plot[smooth] file {figC1.table};
\draw plot[smooth] file {figC2.table};
\smalldot {-1,0} node[anchor=east] {$z_0$};
\smalldot {12/13,5/13} node[anchor=west] {$z_1$};
\smalldot {7/25,24/25} node[anchor=south west] {$z_2$};
\smalldot {-36/325,-323/325} node[anchor=north east] {$z_3$};
}
}

\section{Deforming the Circle}

We can use exactly the same hyperbola construction to define a binary
operation $\oplus$ on other curves.  We call this {\it hyperbolic
  addition} on a curve.  We replace the unit circle with a more
general algebraic curve $C$, defined by the zero set of
\begin{equation}\label{eqn:ed}
x^2 + c y^2 - 1 - d x^2 y^2
\end{equation} for some parameters $c$ and $d$.
This zero locus of this polynomial is called an {\it Edwards curve}.%
\footnote{This definition is more inclusive than definitions stated
  elsewhere.  Most writers prefer to restrict to curves of genus one
  and generally call a curve with $c\ne 1$ a {\it twisted Edwards
    curve}.  We have interchanged the $x$ and $y$ coordinates on the
  Edwards curve to make it consistent with the group law on the
  circle.}  The unit circle corresponds to parameter values $c=1$ and
$d=0$.

 % Fig 2.
\tikzfig{fig2}{The figure on the left is an Edwards curve (with
  parameters $c=0$ and $d=-8$).  An Edwards curve and hyperbola meet
  at four points $z_0 = (-1,0)$, $z_1$, $z_2$, and $z_3$.  By
  construction, hyperbolic addition satisfies $z_1 \oplus z_2 =
  \iota(z_3)$.}
{
{
\begin{scope}[xshift=0]
\draw plot[smooth] file {figE1.table};
\draw plot[smooth] file {figE2.table};
\end{scope}
\begin{scope}[xshift=5cm]
\draw plot[smooth] file {figE1.table};
\draw plot[smooth] file {figE2.table};
\draw plot[smooth] file {figE3.table};
\draw plot[smooth] file {figE4.table};
\smalldot {-1,0} node[anchor=east] {$z_0$};
\smalldot {0.9,0.158} node[anchor=west] {$z_1$};
\smalldot {0.2,0.853} node[anchor=south west] {$z_2$};
\smalldot {0.037,-0.993} node[anchor=north east] {$z_3$};
\end{scope}
}
}

We define a binary operation on the Edwards curve by the hyperbolic
addition law described above.  Let $(-1,0)$, $z_1 = (x_1,y_1)$ and
$z_2=(x_2,y_2)$ be three points on an Edwards curve that are not
collinear (to avoid degenerate cases).  We fit a hyperbola of the
usual form (\ref{eqn:hyp}) through these three points, and let $z_3$
be the fourth point of intersection of the hyperbola with the curve.
We define the hyperbolic sum $z_1\oplus z_2$ of $z_1$ and $z_2$ to be
$\iota(z_3)$.  The following lemma gives an explicit formula for
$z_1\oplus z_2 = \iota(z_3)$.
 
 \begin{lemma}\label{lemma:hyp} 
In this construction, the coordinates are given explicitly by
 \begin{equation}\label{eqn:sum}
 \iota(z_3) = \left(\frac{x_1 x_2 - c y_1 y_2}{1 - d x_1 x_2 y_1 y_2},
\frac{x_1 y_2 + y_1 x_2}{1+d x_1 x_2 y_1 y_2}\right)
 \end{equation}
 \end{lemma}

 This lemma will be proved below (Lemma~\ref{lemma:hyperbola}).  Until
 now, we have assumed the points $(-1,0)$, $z_1$, and $z_2$ are not
 collinear.  Dropping the assumption of non-collinearity, we turn
 formula (\ref{eqn:sum}) of Lemma~\ref{lemma:hyp} into a definition
 and define the hyperbolic sum
\[
 z_1\oplus z_2 := \iota(z_3).
\]
algebraically by that formula in all cases.  We prove below an affine
closure result (Lemma~\ref{lemma:affine}) showing that the
denominators are always nonzero for suitable parameters $c$ and $d$.
In the case of a circle ($c=1$, $d=0$), the formula (\ref{eqn:sum})
reduces to the usual group law (\ref{eqn:cx}).


